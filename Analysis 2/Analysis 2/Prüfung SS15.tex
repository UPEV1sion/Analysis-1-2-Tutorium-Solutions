\documentclass[a4paper, ngerman, 11pt]{article}
\usepackage[T1]{fontenc}
\usepackage[utf8]{inputenc}
\usepackage{babel, lmodern, amsmath, amsfonts, graphicx, amssymb, mathrsfs}
\usepackage[centering, margin={1in, 0.5in}, includeheadfoot]{geometry}

\begin{document}
	\subsubsection*{Aufgabe 1}
	(10 Punkte, ohne Gewähr)
	\vspace*{1cm}
	
	\noindent
	Gegeben ist die Funktion $f(x) = \frac{1-\cos(x)}{x}$.
	\vspace*{0.5cm}
	
	\noindent
	\textbf{a(7)} \quad Bestimmen Sie das Taylorpolynom fünften Grades von f im Entwicklungspunkt~$x_0=0$
	
	\qquad (Sie können zur Lösung bekannte Taylorreihen verwenden.).
	
	\vspace*{1cm}
	
	$\text{Taylorreihe für } \cos(x) := 1 - \frac{x^2}{2!} + \frac{x^4}{4!} -\frac{x^6}{6!} + ...$
	\vspace*{0.5cm}
	
	$\Rightarrow T_{f,0} (x) = \cfrac{1-\left(1 - \frac{x^2}{2!} + \frac{x^4}{4!} -\frac{x^6}{6!}\right)}{x} = \cfrac{\frac{x^2}{2!} - \frac{x^4}{4!} + \frac{x^6}{6!}}{x} = \frac{x}{2!} - \frac{x^3}{4!} + \frac{x^5}{6!}$

	\vspace*{2cm}
	\noindent
	\textbf{b(3)} \quad Bestimmen Sie den Fehler an der Stelle $x = 2\pi$
	
	\vspace*{1cm}
	
	$f(2\pi) = \cfrac{1-cos(2\pi)}{2\pi} = \cfrac{1-1}{2\pi} = 0$
	\vspace*{0.5cm}
	
	\begin{flalign*}
		\qquad T_{f,0}(2\pi) &= \cfrac{2\pi}{2!} - \cfrac{(2\pi)^3}{4!} + \cfrac{(2\pi)^5}{6!}&&\\
		&= \cfrac{2\pi}{2} - \cfrac{8\pi^3}{24} + \cfrac{32\pi^5}{720}&&\\
		&= \pi - \cfrac{\pi^3}{3} + \cfrac{2\pi^5}{45}&&\\
		&= \pi \left( 1 - \cfrac{\pi^2}{3} + \cfrac{2\pi^4}{45}\right)&&\\
	\end{flalign*}
	\newpage
	\subsubsection*{Aufgabe 2}
	
	(15 Punkte, ohne Gewähr)
	
	\bigskip
	\noindent
	Gegeben sei die Funktion $ f : \mathbb{R}^2 \to \mathbb{R} $
	
	\smallskip
	\noindent
	$f(x,y) = 3x^2y + 2y^3 - 6y - 1$
	
	\noindent
	Außerdem sei gegeben der Richtungsvektor $\vec{a} = \frac{\sqrt{2}}{2} \begin{pmatrix} 1 \\ -1 \end{pmatrix}$
	
	\begin{enumerate}
		\item[\textbf{a(7)}] Berechnen Sie den Gradienten $\nabla f(x,y)$ und die Hessematrix $Hf(x,y)$.
		\item[\textbf{b(3)}] Berechnen Sie die Richtungsableitung von $f$ im Punkt $P(-1,1)$ in Richtung $\vec{a}$.
		\item[\textbf{c(5)}] \noindent
		 \quad Bestimmen Sie alle Extrema von $f$ und bestimmen Sie ob an diesen Stellen ein~lokales \\
		\qquad Maximum, ein lokales Minimum, ein Sattelpunkt oder keines der vorhergehenden Möglich-\\ 
		\qquad keiten vorliegt.
	\end{enumerate}

	\subsection*{Lösung}
	\begin{enumerate}
		\item[a)]
		\[
		\frac{\partial f}{\partial x}(x,y) = 6xy
		\]
		\[
		\frac{\partial f}{\partial y}(x,y) = 3x^2 + 6y^2 - 6
		\]
		\[
		\nabla f(x,y) = \begin{pmatrix} 6xy \\ 3x^2 + 6y^2 - 6 \end{pmatrix}
		\]
		\[
		Hf(x,y) = \begin{pmatrix}
			6y & 6x \\
			6x & 12y - 6
		\end{pmatrix}
		\]
		
		\item[b)]
		\[
		\nabla f(-1,1) = \begin{pmatrix} 6(-1)(1) \\ 3(-1)^2 + 6(1)^2 - 6(1) \end{pmatrix} = \begin{pmatrix} -6 \\ 3 + 6 - 6 \end{pmatrix} = \begin{pmatrix} -6 \\ 3 \end{pmatrix}
		\]
		\[
		\nabla f(-1,1) \cdot \vec{a} = \begin{pmatrix} -6 \\ 3 \end{pmatrix} \cdot \frac{\sqrt{2}}{2} \begin{pmatrix} 1 \\ -1 \end{pmatrix} = \frac{\sqrt{2}}{2} \begin{pmatrix} -6 \\ 3 \end{pmatrix} \cdot \begin{pmatrix} 1 \\ -1 \end{pmatrix} = \frac{\sqrt{2}}{2} (-6 \cdot 1 + 3 \cdot (-1)) = \frac{\sqrt{2}}{2} (-6 - 3) = -\frac{9\sqrt{2}}{2}
		\]
		
		\item[c)] Extrema $\Rightarrow \nabla f(x,y) = 0$
		\[
		\Rightarrow 6xy = 0 \quad \Rightarrow x = 0 \, \text{oder} \, y = 0
		\]
		\[
		\text{für } x = 0\Rightarrow 6y^2 - 6 = 0 \Rightarrow y^2 = 1 \quad \Rightarrow y_{1,2} = \pm 1
		\]
		\[
		\text{für } y = 0 \Rightarrow 3x^2 -6 = 0 \Rightarrow x^2 = 2 \quad \Rightarrow x_{1,2} = \pm\sqrt{2}
		\]
		
		\[
		P_1(0,1), P_2(0,-1), P_3(\sqrt{2},0), P_4(-\sqrt{2},0)
		\]
		
		\[
		Hf(0,1) = \begin{pmatrix} 6 & 0 \\ 0 & 12 \end{pmatrix} \, \text{positiv definit} \Rightarrow \text{Minimum}
		\]
		
		\[
		Hf(0,0) = \begin{pmatrix} -6 & 0 \\ 0 & -12 \end{pmatrix} \, \text{negativ definit} \Rightarrow \text{Maximum}
		\]
	\end{enumerate}
	
	
	
\end{document}