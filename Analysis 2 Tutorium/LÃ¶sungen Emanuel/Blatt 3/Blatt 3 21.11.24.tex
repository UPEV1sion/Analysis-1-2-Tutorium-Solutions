\documentclass[ngerman, a4paper]{scrartcl}
\usepackage{math_emanuel}

\begin{document}
	\title{Lösungen der Übungsaufgaben Analysis 2 für den 21.11.24}
	\author{Emanuel Schäffer}
	\maketitle
	
	\section*{Aufgabe 1}
	Beweisen Sie den folgenden Satz: Eine Vektorfolge $(\vec{a}_n)_{n\in \mathbb{N}}$ konvergiert genau dann gegen~$\vec{a}$, wenn alle ihre Koordinatenfolgen gegen entsprechende Koordinaten von $\vec{a}$ konvergieren.
	
	\vspace{0.5cm}
	\subsection*{Lösung Aufgabe 1:}
	
	\noindent Konvergiert also die Vektorfolge $(\vec{a}_n)_{n\in \mathbb{N}}$ gegen $\vec{a}$
	dann gilt gemäß Definition:
	\[\norm{\vec{a}_n - \vec{a}} < \varepsilon \quad \forall \quad n \ge N(\varepsilon)\]
	
	also $\lim\limits_{n\rightarrow\infty} \vec{a}_n = \vec{a}$ beziehungsweise $\lim\limits_{n\rightarrow\infty} \norm{\vec{a}_n - \vec{a}} = 0$ 
	
	\noindent Wegen der Definition der Norm folgt aus $\norm{\vec{a}_n - \vec{a}} = 0$ das alle Komponenten des Vektors $\vec{a}_n - \vec{a}$ Null sein müssen. Also konvergieren alle Komponenten der Folge $\vec{a}_n$ gegen die entsprechende Komponente von $\vec{a}$
	
	\vspace{0.5cm}
	
	Konvergiert umgekehrt jede Komponente $i$ der Folge $\vec{a}_n$ gegen die entsprechende Komponente von  $\vec{a}$ sprich $a_i$ so gilt wegen $\lim\limits_{n\rightarrow\infty}|a_i^{(n)} - a_i| = 0$ 
	
	\noindent dass 
	$\lim\limits_{n\rightarrow\infty}\norm{\vec{a}_n - \vec{a}} = 
	\inflimit \sqrt{(a_1^ {(n)} - a_1)^2 + (a_2^ {(n)} - a_2)^2 + \ldots} = 0$ also konvergiert die Vektorfolge $(\vec{a}_n)_{n\in\mathbb{N}}$ gegen $\vec{a}$

	
	\section*{Aufgabe 2}
	\begin{samepage}
	\begin{enumerate}[\textbf{\alph*)}]
		\item In der industriellen Fertigung werden bei der Qualitätskontrolle Bauteile vermessen und die Werte $ x_1, \ldots, x_n $ ermittelt. Der Vektor $\vec{d} = \vec{x} - \vec{s}$ gibt die Abweichungen der Messungen zu den Sollwerten $s_1, \ldots, s_n$ an. Definieren Sie nun eine Norm auf $\mathbb{R}^n$ derart, daß $\vec{\norm{d}} < \varepsilon$ gdw. alle Abweichungen vom Sollwert kleiner als eine gegebene Toleranz $\varepsilon$ sind.
		\item Beweisen Sie, daß die in a) definierte Norm alle Axiome einer Norm erfüllt.
	\end{enumerate}	
	\end{samepage}
	
	\subsection*{Lösung Aufgabe 2}
	
	\begin{enumerate}[\textbf{\alph*)}]
		\item Die in Frage kommende Abbildung ist die sogenannte Maximum-Norm:
		
		$\norm{\vec{x}}_{max} \coloneqq \max(|x_1|, \dots, |x_n|)$
		\item Die Abbildung $\norm{\quad}_{max}: \mathbb{R}^n \rightarrow \mathbb{R}$ ist eine Norm wegen:
		\begin{enumerate}[\arabic*.]
			\item Aus $\norm{\vec{x}}_{max} = \max(|x_1|, \dots, |x_n|) = 0$ folgt $x_1 = x_2 = \ldots = x_n = 0$
			\item Es gilt $\norm{\lambda \cdot \vec{x}}_{max} = \max(|\lambda \cdot x_1|, \dots, |\lambda \cdot x_n|) = |\lambda| \cdot \max(|x_1|, \dots, |x_n|) = |\lambda| \cdot \norm{\vec{x}}_{max} \forall \lambda \in \mathbb{R}, \vec{x} \in \mathbb{R}^n$ 
			\item Es gilt $\norm{\vec{x} + \vec{y}}_{max} = \max(|x_1 + y_1|, \ldots, |x_n + y_n|) \le \max(|x_1| + |y_1|, \ldots, |x_n| + |y_n|) \le \max(|x_1|, \ldots, |x_n|) + \max(|y_1|, \ldots, |y_n|) = \norm{\vec{x}}_{max} + \norm{\vec{y}}_{max} \forall x, y \in \mathbb{R}^n$ Dreiecksungleichung
		\end{enumerate}
	\end{enumerate}
	
	
	\section*{Aufgabe 3}
	Berechnen Sie die partiellen Ableitungen $\frac{\partial f}{\partial x_1}$, $\frac{\partial f}{\partial x_2}$, $\frac{\partial f}{\partial x_1}$ folgender Funktionen $f: \mathbb{R}^3\rightarrow\mathbb{R}$
		
	\begin{enumerate}[\textbf{\alph*)}]
		\item $f(\vec{x}) = |\vec{x}|$
		\item $f(\vec{x}) = x_1^{x_2} + x_1^{x_3}$
		\item $f(\vec{x}) = x_1^{(x_2 + x_3)}$
		\item $f(\vec{x}) = \sin(x_1 + x_2)$
		\item $f(\vec{x}) = \sin(x_1 + a\cdot x_2)$
	\end{enumerate}
	
	\subsection*{Lösung Aufgabe 3}
	
	\begin{enumerate}[\textbf{\alph*)}]
		\item $f(\vec{x}) = |\vec{x}| = \sqrt{x_1^2 + x_2^2 + x_3^2}$\\
		$\partder{x_1} = \frac{1}{2} \cdot (x_1^2 + x_2^2 + x_3^2)^{-\frac{1}{2}}\cdot 2x_1 = \frac{x_1}{|\vec{x}|}$\\
		$\partder{x_2} = \frac{x_2}{|\vec{x}|}$\\
		$\partder{x_3} = \frac{x_3}{|\vec{x}|}$\\
		
		\item $ f(\vec{x}) = x_1^{x_2} + x_1^{x_3}  = e^{\ln(x_1^{x_2})} + e^{\ln(x_1^{x_3})} = e^{x_2 \cdot\ln(x_1)} + e^{x_3 \cdot \ln(x_1)} $\\
		$\partder{x_1} = x_2 \cdot x_1^{x_2-1} + x_3\cdot x_1^{x_3-1}$\\
		$\partder{x_2} = e^{x_2 \cdot \ln(x_1)} \cdot \ln(x_1) = x_1^{x_2}\cdot \ln(x_1)$\\
		$\partder{x_3} = x_1^{x_3}\ln(x_1)$
		
		\item $ f(\vec{x}) = x_1^{(x_2 + x_3)} = x_1^{x_2} \cdot x_1^{x_3} = e^{x_2\ln(x_1)} \cdot e^{x_3\ln(x_1)} $\\
		$\partder{x_1} = (x_2 + x_3)\cdot x_1^{(x_2 + x_3)-1}$\\
		$\partder{x_2} = e^{x_2\ln(x_1)} \cdot e^{x_3\ln(x_1)} \cdot \ln(x_1) = x_1^{(x_2 + x_3)} \cdot \ln(x_1)$\\
		$\partder{x_3} = x_1^{(x_2 + x_3)} \cdot \ln(x_1)$
		
		\item $f(\vec{x}) = \sin(x_1 + x_2)$\\
		$\partder{x_2} = \cos(x_1 + x_2) \cdot 1$\\
		$\partder{x_1} = \cos(x_1 + x_2) \cdot 1$\\
		$\partder{x_3} = 0$
		 
		\item $f(\vec{x}) = \sin(x_1 + a\cdot x_2)$\\
		$\partder{x_1} = \cos(x_1 + a\cdot x_2) \cdot 1$\\
		$\partder{x_2} = \cos(x_1 + a\cdot x_2) \cdot a$\\
		$\partder{x_3} = 0$\\
	\end{enumerate}
	
	\section*{Aufgabe 4}
	Berechnen Sie die Ableitungsmatrix der Funktion $\vec{f}(x_1, x_2, x_3) = \begin{pmatrix}
		\sqrt{x_1x_2x_3}\\
		\sin(x_1x_2x_3)
	\end{pmatrix}$.
	
	\subsection*{Lösung Aufgabe 4}
	\[
	d\vec{f}(x_1, x_2, x_3) =
	\begin{pmatrix}
		\frac{x_2x_3}{2\sqrt{x_1x_2x_3}}, & \frac{x_1x_3}{2\sqrt{x_1x_2x_3}}, & \frac{x_1x_2}{2\sqrt{x_1x_2x_3}}\\
		\cos(x_1x_2x_3)x_2x_3, & \cos(x_1x_2x_3)x_1x_3, & \cos(x_1x_2x_3)x_1x_2
	\end{pmatrix}
	\]
	
	\section*{Aufgabe 5}
	Geben Sie für 
	$
	\vec{f}(x, y) = 
	\begin{pmatrix}
		\sqrt{xy}\\
		\sin(e^x + e^y)
	\end{pmatrix}
	$
	die Tangentialebene im Punkt 
	$
	\vec{x_0} = 
	\begin{pmatrix}
		1\\
		2
	\end{pmatrix}
	$
	an.
	
	\subsection*{Lösung Aufgabe 5}
	\[
		d\vec{f}(x, y) = 
		\begin{pmatrix}
			\frac{y}{2\sqrt{xy}}, & \frac{x}{2\sqrt{xy}}\\
			\cos(e^x + e^y) \cdot e^x, & \cos(e^x + e^y) \cdot e^y
		\end{pmatrix}
	\]
	Die Funktion der Tangentialebene lautet:
	\[
		\vec{g}(x, y) = \vec{f}(x_0, y_0) + d\vec{f}(x_0, y_0) \cdot
		\begin{pmatrix}
			x-x_0\\
			y-y_0
		\end{pmatrix}
	\]
	Mit $x_0 = 1 \text{ und } y_0 = 2$ folgt:
	
	\[
		\vec{g}(x, y) =
		\begin{pmatrix}
			\sqrt{2}\\
			\sin(e^1 + e^2)
		\end{pmatrix}
		+
		\begin{pmatrix}
			\frac{2}{2\sqrt{2}}, & \frac{1}{2\sqrt{2}}\\
			\cos(e^1 + e^2) \cdot e^1, & \cos(e^1 + e^2) \cdot e^2
		\end{pmatrix}
		\cdot
		\begin{pmatrix}
			x-1\\
			y-2
		\end{pmatrix}
	\]
	
	\[
		=
		\begin{pmatrix}
			\sqrt{2} + \frac{1}{\sqrt{2}}\cdot(x-1) + \frac{1}{2\sqrt{2}}\cdot(y-2)\\
			\sin(e(1 + e)) + \cos(e(1 + e))e\cdot(x-1) + \cos(e(1 + e))e^2\cdot(y-2)
		\end{pmatrix}
	\]
	\[
		=
		\begin{pmatrix}
			\frac{1}{\sqrt{2}}\cdot(2+x-1+\frac{y}{2}-1)\\
			\sin -e\cdot\cos + e\cdot x\cdot \cos -2e^2\cdot \cos + y \cdot e^2 \cos
		\end{pmatrix}
	\]
	\[
	=
	\begin{pmatrix}
		\frac{1}{\sqrt{2}}\cdot(x+\frac{y}{2})\\
		\sin + \cos\left[-e - 2e^2 + x\cdot e + y\cdot e^2 \right] 
	\end{pmatrix}
	\]
	
\end{document}