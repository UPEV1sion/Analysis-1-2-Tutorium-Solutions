\documentclass[ngerman, a4paper]{scrartcl}
\usepackage[utf8]{inputenc}
\usepackage[T1]{fontenc}
\usepackage[ngerman]{babel}
\usepackage{lmodern}
\usepackage{mathtools, amsfonts, amssymb}
\usepackage{graphicx}
\usepackage{tikz}
\usepackage{xcolor}
\usepackage{pgfplots}
\usepackage{animate}
\usepackage[shortlabels]{enumitem}
\pgfplotsset{width=7cm, compat=1.18}
\usetikzlibrary{patterns, positioning, shapes.geometric}

\setlength\parindent{0pt}

\DeclareMathOperator{\Hessian}{Hess}

\newcommand{\norm}[1]{\left\lVert#1\right\rVert}
\newcommand{\inflimit}{\lim\limits_{n\rightarrow\infty}}
\newcommand{\partderf}[1]{\tfrac{\partial f}{\partial #1}}
\newcommand{\partder}[2]{\tfrac{\partial #1}{\partial #2}}
\newcommand{\zpartderf}[2]{\tfrac{\partial^2 f}{\partial #1 \partial #2}}
\newcommand{\zpartder}[3]{\tfrac{\partial^2 #1}{\partial #2 \partial #3}}

\begin{document}
	\title{Lösungen der Übungsaufgaben Analysis 2 für den 5.12.24}
	\author{Emanuel Schäffer}
	\maketitle
	
	\section*{Aufgabe 1}
	Untersuchen Sie folgende Funktionen auf Extrema und geben Sie an, ob es sich um ein lokales oder globales, bzw. ein isoliertes Extremum handelt:
	
	\begin{align*}
		\textbf{a)} \quad f(x,y) &= x^3y^2(1-x-y)\\
		\textbf{b)} \quad g(x,y) &= x^k+(x+y)^2 \qquad (k=0,3,4)
	\end{align*}
	
	\subsection*{Lösung Aufgabe 1}
	\begin{tikzpicture}
		\begin{axis}[
			title={$f(x,y) = x^3y^2(1-x-y)$},
			xlabel={$x$},
			ylabel={$y$}, 
			zlabel={$z$}
			]
			\addplot3[mesh, domain=-10:10, samples=50] {x^3 * y^2 * (1-x-y)};
		\end{axis}
	\end{tikzpicture}
	\animategraphics[loop, controls, width=7cm]{0.5}{frame-}{1}{3}
	
	\newpage
	\begin{enumerate}[\textbf{\alph*)}]
		\item 
		$
		f(x,y) = x^3y^2(1-x-y) = x^3y^2 - x^4y^2 - x^3y^3 \\
		\partderf{x}(x,y) = 3x^2y^2 - 4x^3y^2 - 3x^2y^3 \\
		\partderf{y}(x,y) = 2x^3y - 2x^4y - 3x^3y^2 \\
		\zpartderf{x}{x}(x,y) = 6xy^2 - 12x^2y^2 - 6xy^3 \\
		\zpartderf{y}{y} (x,y) = 2x^3 - 2x^4 - 6x^3y \\
		\zpartderf{x}{y} (x,y) = 6x^2y - 8x^3y - 9x^2y^2 
		$
%	\textbf{a)}
%	\begin{align*}
%		f(x,y) &= x^3y^2(1-x-y) = x^3y^2 - x^4y^2 - x^3y^3 &&\\
%		\partderf{x}(x,y) &= 3x^2y^2 - 4x^3y^2 - 3x^2y^3 &&\\
%		\partderf{y}(x,y) &= 2x^3y - 2x^4y - 3x^3y^2 &&\\
%		\zpartderf{x}{x}(x,y) &= 6xy^2 - 12x^2y^2 - 6xy^3 &&\\
%		\zpartderf{y}{y} (x,y) &= 2x^3 - 2x^4 - 6x^3y &&\\
%		\zpartderf{x}{y} (x,y) &= 6x^2y - 8x^3y - 9x^2y^2 &&
%	\end{align*}

		Notwendige Bedingung für ein Extremum ist, dass $\nabla f = 0$ also
		
		\begin{align*}
			3x^2y^2 - 4x^3y^2 - 3x^2y^3 &= 0 = x^2y^2(3-4x-3y) \text{\quad und} &&\\
			2x^3y - 2x^4y - 3x^3y^2 &= 0 = x^3y(2-2x-3y)&&
		\end{align*}
		
		also 1. Lösung: $x = 0$, $y$ beliebig ($y$-Achse)\\
		und 2. Lösung. $y = 0$,  $x$ beliebig ($x$-Achse)\\
		
		Außerdem $3 - 4x - 3y = 0$ und $2 - 2x - 3y = 0$ also $1 - 2x = 0$ also 3. Lösung: $x = \frac{1}{2}$ und $y = \frac{1}{3}$
		
		\[
			\Hessian f(x, y) =
			\begin{pmatrix}
				6xy^2 - 12x^2y^2 - 6xy^3 & 6x^2y - 8x^3y - 9x^2y^2 \\
				6x^2y - 8x^3y - 9x^2y^2 & 2x^3 - 2x^4 - 6x^3y
			\end{pmatrix}
			\Rightarrow
		\]
		
		\[
			\Hessian f(0, y) = 
			\begin{pmatrix}
				0 & 0\\
				0 & 0
			\end{pmatrix}
			\Rightarrow \text{semidefinit}
		\]
		
		\[
			\Hessian f(x, 0) =
			\begin{pmatrix}
				0 & 0\\
				0 & 2x^3 - 2x^4
			\end{pmatrix}
			\Rightarrow
		\]
		
		In Abhängigkeit von $x$ positiv oder negativ semidefinit;
		
		positiv für $0 < x < 1$ und negativ für $x < 0$ oder $x > 1$
		
		\[
			\Hessian f(\frac{1}{2}, \frac{1}{3}) = 
			\begin{pmatrix}
				-\frac{1}{9} & -\frac{1}{12} \\
				-\frac{1}{12} & -\frac{1}{9}
			\end{pmatrix}
			\Rightarrow
		\]
		$\Hessian f(\frac{1}{2}, \frac{1}{3})$ ist negativ definit; also lokales Maximum
		\item 
		$
		g(x,y) = x^k+(x+y)^2 \qquad (k=0,3,4)\\
		\partder{g}{x}(x,y) = k\cdot x^{k-1} + 2(x + y)\\
		\partder{g}{y}(x,y) = 2(x + y)
		$
		\\
		
		
		\begin{samepage}
			\textbf{Fall $k = 0$}\\
			Wegen $g(x,y) = 1 + (x + y)^2 \ge 1$ und $g(x,y) = 1$ für $x + y = 0$\\
			$\Rightarrow$ in allen Punkten der Gerade $x + y = 0$ bzw. $y = -x$ hat $g$ ein Minimum.
		\end{samepage}
	
		\textbf{Fall $k = 3,4$}
		
		Aus der Bedingung $\nabla f = 0$ folgt:\\
		
		$2(x + y) = 0$ also $y = -x$ und\\
		$k\cdot x^{k-1} + 2(x + y) = k\cdot x^{k-1} + 2(x - x) = k\cdot x^{k-1} = 0 \quad \Rightarrow \quad  x = 0 \text{ und } y = 0$
		\[
			\Hessian f(0,y) = 
			\begin{pmatrix}
				2 & 2\\
				2 & 2
			\end{pmatrix}
			\Rightarrow \text{semidefinit}
		\]
		
		\textbf{Fall $k = 3$}
		
		$g(x, -x) = x^3 \quad \Rightarrow \quad$ Wendepunkt\\
		
		\textbf{Fall $k = 4$}
		
		$g(x,y) = x^4 + (x + y)^2 > 0 \text{ für } (x,y) \ne (0,0)$ und\\
		$g(x,y) = x^4 + (x + y)^2 = 0 \text{ für } (x,y) = 0 \Rightarrow$ lokales Minimum
		
		\section*{Aufgabe 2}
		Gegen sie die Funktion $f: \mathbb{R}^2 \rightarrow \mathbb{R}, f(x,y) = (y-x^2)(y-3x^2).$
		\begin{enumerate}[\textbf{\alph*)}]
			\item Berechnen Sie $\nabla f$ und zeigen Sie: $\nabla f(x,y) = 0 \Leftrightarrow x = y = 0$.
			\item Zeigen Sie, daß $(\Hessian f)(0)$ semidefinit ist und daß $f$ auf jeder Geraden durch~0 ein isoliertes Minimum hat.
			\item Trotzdem hat $f$ in 0 kein lokales Extremum (zu zeigen!).
		\end{enumerate}
	
		\subsection*{Lösung Aufgabe 2}
		\begin{enumerate}[\textbf{\alph*)}]
			\item 
			$
			f(x,y) = (y-x^2)(y-3x^2) = 3x^4 - 4x^2y + y^2\\
			\partderf{x}(x,y) = 12x^3 - 8xy = 4x(3x^2 - 2y)\\
			\partderf{y}(x,y) = -4x^2 + 2y = 2(y - 2x^2)
			$
			\\
			
			Die Bedingung $\nabla f = 0$ führt auf $y - 2x^2 = 0$ bzw. $y = 2x^2$\\
			und auf $4x(3x^2 - 2y)  = 4x(3x^2 - 4x^2) = -4x^3 = 0$\\
			$\Rightarrow \quad x = 0 \text{ und } y = 0$
			
			\item 
			Zeigen Sie, daß $(\Hessian f)(0)$ semidefinit ist und daß $f$ auf jeder Geraden durch 0 ein isoliertes Minimum hat.
			
			\[
				\Hessian f(x,y) = 
				\begin{pmatrix}
					36x^2 - 8y & -8x\\
					-8x & 2
				\end{pmatrix}
				\Rightarrow
			\]
			
			\[
				\Hessian f(0,0) = 
				\begin{pmatrix}
					0 & 0\\
					0 & 2
				\end{pmatrix}
				\quad \Rightarrow \text{positiv semidefinit}
			\]
			Betrachte $f$ auf der Ortskurve $y = \alpha \cdot x $ mit $\alpha \in \mathbb{R}$ (Alle Geraden durch den Nullpunkt)
			
			$
			\Rightarrow \phi(x) = f(x, \alpha x) = (\alpha x-x^2)(\alpha x - 3x^2) = 3x^4 - 4\alpha x^3 + \alpha^2 x^2\\
			\phi'(x) = 12x^3 - 8\alpha x^2 + 2\alpha^2x \quad \Rightarrow \phi^\prime(x) = 0\\
			\phi''(x) = 36x^2 - 16\alpha x + 2\alpha^2 \quad \Rightarrow \phi''(0) > 0
			$
			
			Also hat $f$ auf der Geraden $y = \alpha x $ in $(0,0)$ ein Minimum
			
			\item Trotzdem hat $f$ in 0 kein lokales Extremum (zu zeigen!).\\
			Betrachte $f$ auf der Ortskurve $y = 2x^2$
			\\
			
			$\Rightarrow f(x, 2x^2) = x^2(-x^2)$
			\\
			
			und betrachte $f$ auf der Ortskurve $y = 0$ ($x$-Achse) 
			\\
			
			$\Rightarrow f(x, 0) = 3x^4$
			Also hat $f$ in $(0,0)$ kein lokales Extremum (Sattelpunkt)
		\end{enumerate}
	
		\section*{Aufgabe 3}
		Gegen sie die Funktion $f: \mathbb{R}^2 \rightarrow \mathbb{R}$,
		
		\[
			f(x,y) = 
			\begin{cases}
				\tfrac{xy^3}{x^2 + y^6} &\text{ für } (x,y) \ne (0,0)\\
				\quad 0 &\text{ für } (x,y) = (0,0)
			\end{cases}
		\]
		
		Zeigen Sie, daß im Punkt $(0,0)$ alle Richtungsableitungen existieren.
		Ist $f$ in $(0,0)$ differenzierbar?
		
		(Hinweis: Diese Aufgabe ist nicht ganz einfach. Betrachten Sie einen Kreis um den Nullpunkt und lassen Sie den den Radius des Kreises gegen Null  streben).
		
		\subsection*{Lösung Aufgabe 3}
		
		Gegeben sei ein Kreis um den Nullpunkt $(0,0)$ in $\mathbb{R}^2$ durch $(u,v)\in \mathbb{R}^2$ mit $u^2~+~v^2~=~1$ \\
		Betrachte nun $f$ auf dem Kreis um den Nullpunkt mit dem Kreisradius $\rightarrow 0$
		
		Allgemein gilt daher:
		
		\[
			\lim\limits_{t\rightarrow 0} \frac{f(t\cdot u, t\cdot v) - f(0,0)}{t} = 
			\lim\limits_{t \rightarrow 0}\frac{t\cdot u\cdot(t\cdot v)^3}{t\cdot \left[ (t\cdot u)^2 + (t\cdot v)^6\right]} = 
			\lim\limits_{t \rightarrow 0} t\cdot \frac{u\cdot v^3}{u^2 + t^4\cdot v^6} = 0
		\]
		sowohl für $u \ne 0$ da $\left|\tfrac{u\cdot v^3}{u^2 + t^4\cdot v^6}\right| < \left|\tfrac{u\cdot v^3}{u^2}\right| < $ Konstante, als auch $u = 0$ (und $v = 1$)
		
		Daher existieren alle Richtungsableitungen in $(0,0)$ und haben den Wert 0.
		
		$f$ ist aber in $(0,0)$ nicht total differenzierbar:
		
		
		wegen $f(y^3,y) = \frac{y^6}{y^6 + y^6} = \frac{1}{2}$ für $y \rightarrow 0$ und $f(0,0) = 0$\\
		ist $f$ an der Stelle $(0,0)$ nicht stetig, also an der Stelle $(0,0)$ nicht total differenzierbar.
	\end{enumerate}
\end{document}
