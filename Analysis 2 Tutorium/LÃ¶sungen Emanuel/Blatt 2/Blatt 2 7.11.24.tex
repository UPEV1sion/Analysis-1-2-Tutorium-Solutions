\documentclass[a4paper, ngerman]{scrartcl}
\usepackage[utf8]{inputenc}
\usepackage[T1]{fontenc}
\usepackage{babel}
\usepackage{lmodern}
\usepackage{amsmath, amsfonts, amssymb}
\usepackage{xcolor}
\usepackage{listings}

\definecolor{main-color}{rgb}{0.6627, 0.7176, 0.7764}
\definecolor{back-color}{rgb}{0.1686, 0.1686, 0.1686}
\definecolor{string-color}{rgb}{0.3333, 0.5254, 0.345}
\definecolor{key-color}{rgb}{0.13,0.13,1}
\definecolor{typedef-color}{rgb}{0.8, 0.47, 0.196}

\lstdefinestyle{mystyle}
{
	language = C,
	basicstyle = {\ttfamily},
	showspaces = false,
	showstringspaces = false,
	%	backgroundcolor = {\color{back-color}},
	stringstyle = {\color{string-color}},
	keywordstyle = {\color{key-color}},
	keywordstyle = [3]{\color{typedef-color}},
	keywordstyle = [4]{\color{teal}},
	morekeywords = [3]{uint8_t, bool},
}


\begin{document}
	\title{Übungsaufgaben Analysis 2 für die Übungen am 7.11.24}
	\author{Emanuel Schäffer}
	\maketitle
	
	\section*{Aufgabe 1}
	\subsection*{a)}
	Für eine gerade Funktion d.h. $f(x) = f(-x)$ kommen in einer Taylorreihe nur gerade Exponenten vor:
	
	$$\sum_{k=0}^{\infty} a_{2k}x^{2k}$$
	
	\begin{align*}
		f(x) &= \sum_{k=0}^{\infty} a_kx^k = \sum_{k=0}^{\infty} a_{2k}x^{2k} + \sum_{k=0}^{\infty} a_{2k + 1}x^{2k + 1}\\
		f(-x) &= \sum_{k=0}^{\infty} a_k(-x)^k = \sum_{k=0}^{\infty} a_{2k}x^{2k} - \sum_{k=0}^{\infty} a_{2k + 1}x^{2k + 1}\\
	\end{align*}
	
	$f(x) = f(-x)$ liefert:
	\begin{align*}
		\sum_{k=0}^{\infty} a_{2k}x^{2k} + \sum_{k=0}^{\infty} a_{2k + 1}x^{2k + 1} &= \sum_{k=0}^{\infty} a_{2k}x^{2k} - \sum_{k=0}^{\infty} a_{2k + 1}x^{2k + 1} \qquad |-\sum_{k=0}^{\infty} a_{2k}x^{2k} \\
		\sum_{k=0}^{\infty} a_{2k + 1}x^{2k + 1} &= - \sum_{k=0}^{\infty} a_{2k + 1}x^{2k + 1}
	\end{align*}
	$\displaystyle{\sum_{k=0}^{\infty} a_{2k + 1}x^{2k + 1} = 0} \Rightarrow \forall k a_{2k + 1} = 0$ 
	
	\newpage
	\subsection*{b)}
	Für eine gerade Funktion d.h. $f(x) = -f(-x)$ kommen in einer Taylorreihe nur ungerade Exponenten vor:
	
	$$\sum_{k=0}^{\infty} a_{2k+1}x^{2k+1}$$
	
	
	\begin{align*}
		f(x) &= \sum_{k=0}^{\infty} a_kx^k = \sum_{k=0}^{\infty} a_{2k}x^{2k} + \sum_{k=0}^{\infty} a_{2k + 1}x^{2k + 1}\\
		-f(-x) &= -\sum_{k=0}^{\infty} a_k(-x)^k = -\sum_{k=0}^{\infty} a_{2k}x^{2k} + \sum_{k=0}^{\infty} a_{2k + 1}x^{2k + 1}\\
	\end{align*}
	
	$f(x) = -f(-x)$ liefert:
	\begin{align*}
		\sum_{k=0}^{\infty} a_{2k}x^{2k} + \sum_{k=0}^{\infty} a_{2k + 1}x^{2k + 1} &= -\sum_{k=0}^{\infty} a_{2k}x^{2k} + \sum_{k=0}^{\infty} a_{2k + 1}x^{2k + 1} \qquad |-\sum_{k=0}^{\infty} a_{2k+1}x^{2k+1} \\
		\sum_{k=0}^{\infty} a_{2k}x^{2k} &= - \sum_{k=0}^{\infty} a_{2k}x^{2k}
	\end{align*}
	
	$\displaystyle{\sum_{k=0}^{\infty} a_{2k}x^{2k} = 0} \Rightarrow \forall k a_{2k} = 0$ 
	
	\section*{Aufgabe 2}
	\textbf{arctan ableiten:}
	\begin{align*}
		f(x) &= \arctan(x) \qquad |\tan\\
		\tan(f(x)) &= x
	\end{align*}
	\textbf{Einheitskreis:}
	\begin{align*}
		\tan(f(x)) &= \frac{\sin(f(x))}{\cos(f(x))} = x \qquad | \prime\\
		\frac{(\cos^2(f(x)) + \sin^2f(x))\cdot f^\prime(x)}{\cos^2(f(x))} &= 1\\
		\frac{f^\prime(x)}{\cos^2(f(x))} &= 1\\
		f^\prime(x) &= \cos^2(f(x))
	\end{align*}
	\textbf{Pythagoras Identität:}
	\begin{align*}
		\cos^2(f(x)) + \sin^2(f(x)) &= 1\\
		\cos^2(f(x)) + (x\cos(f(x)))^2 &= 1\\
		\cos^2(f(x))(1+x^2) &= 1\\
		\cos^2(f(x)) = \frac{1}{1 + x^2}
	\end{align*}
	\textbf{Ende Einheitskreis:}
	$$f^\prime(x) = \cos^2(f(x)) = \frac{1}{1 + x^2}$$
	\textbf{N-te Ableitung siehe Tabelle Löhmann}	
	
	\noindent
	\textbf{Taylorreihe Arcus Tangens}
	$$T_{\arctan, 0}(x) = x - \frac{x^3}{3} + \frac{x^5}{5} - \frac{x^7}{7} + O(x)$$
	$$T_{\arctan, 0}(x) = 1 - \frac{1}{3} + \frac{1}{5} - \frac{1}{7} + \frac{1}{9} \pm \dots$$
	\textbf{Summe}
	$$\sum_{k=0}^{\infty}(-1)^{k}\frac{x^{2k+1}}{2k + 1}$$
	\textbf{Aprroximation $\tan(\pi/4) = 1$}
	$$n = 23: 4T_{\arctan, 0}(1) = 3.0584$$
	$$n = 25: 4T_{\arctan, 0}(1) = 3.2184$$
	$$\frac{3.0584 + 3.2184}{2} = 3.1384$$
	
	\textbf{C Code:}
\begin{lstlisting}[style=mystyle]
#include <stdio.h>
#include <math.h>
	
static long double calc_taylor_arctan(const int x, const int n)
{
	long double approx = 0;
	for(int k = 0; k <= n; ++k)
	{
		approx += pow(-1, k) * (pow(x, 2*k + 1) / (2*k + 1));
	}
	
	return approx;
}
	
	
	
int main(void)
{
	long double n_24_arctan = 4 * calc_taylor_arctan(1, 24);
	long double n_25_arctan = 4 * calc_taylor_arctan(1, 25);
	printf("4 * taylor arctan, n = 24 at x = 1: %Lf\n", n_24_arctan);
	printf("4 * taylor arctan, n = 25 at x = 1: %Lf\n", n_25_arctan);
	printf("Mean: %Lf\n", (n_24_arctan + n_25_arctan) / 2);
	printf("PI: %f\n", M_PI);
		
	return 0;
}
	\end{lstlisting}
	
	\section*{Aufgabe 3}
	\textbf{Siehe Lösung Löhmann}
	
	\section*{Aufgabe 4}
	\textbf{Siehe Lösung Löhmann}
	
	
	
	
\end{document}